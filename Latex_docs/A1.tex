%https://www.overleaf.com/9637873877xxqppwwxbdwv#6dc7ad

% This is a paper template using the LLNCS macro package for Springer Computer Science proceedings;
% Version 2.20 of 2017/10/04
%
\documentclass[runningheads]{llncs}
%
% A lot of package loading
\usepackage[pdftex]{graphicx}
\usepackage{geometry}
\usepackage[cmex10]{amsmath}
\usepackage{array, algpseudocode}
\usepackage{amsmath, amssymb, amsfonts, parskip, graphicx, verbatim}
\usepackage{url, hyperref}
\usepackage{bm, rotating, adjustbox, latexsym}
\usepackage{tabularx, booktabs}
\newcolumntype{Y}{>{\centering\arraybackslash}X}
\usepackage{float, setspace, mdframed}
\usepackage{color, contour, placeins, subfig, cite}
\usepackage[mathscr]{euscript}
\usepackage[osf]{mathpazo}
\usepackage{pgf, tikz, microtype, algorithm}
\usetikzlibrary{shapes,backgrounds,calc,arrows}
\usepackage{xcolor, colortbl, dsfont}
\usepackage{xspace}



\begin{document}
%
\title{NaCo 24/25 assignment 1 report}

\author{First and last name of each student in the group}
%
\authorrunning{NaCo 24/25}

\institute{Group number $999$}
%
\maketitle % typeset the header of the contribution
%

\begin{abstract}
This document contains the instructions and the format for the report required for submission of the practical assignment for the course Natural Computing. 
\end{abstract}

\section{Introduction}
This document serves as a \textit{description of the practical assignment} for the course Natural Computing in the academic year 2024/2025. In this assignment, you will go on a trip to visit Europe's capital cities. You will start your trip in Leiden, where you will also return. You will travel via helicopter. As this is not the most economical way to commute, you'll want to travel via the shortest route that visits all cities. You note this is an instance of the Travelling Salesman Problem (TSP), and to \emph{heuristically} compute such a route, you will be using a Genetic Algorithm (GA). 

In this assignment, you will work with Python and Latex. You will be asked to provide the following:
\begin{itemize}
   \item An implementation of a Genetic Algorithm designed to solve TSP. You will be provided with an implementation of the problem in Python as supplementary material. 
   \item A Random Search (RS) implementation designed to solve a TSP. 
   \item A report, written and formatted as a scientific paper, using this template, containing:
\begin{itemize}
    \item A clear description of the GA, focussing on the operators implemented. You will implement at least \textbf{two} different mutation operators. 
    \item A literature review of a paper applying a GA to a TSP.
    \item Experimentation with the two different mutation operators, compared to RS, for solving the TSP.  
    \item Discussion of and visualization of the results.
\end{itemize}
\end{itemize}

To help structure your report, we provide a \textit{brief report outline} in this document. Please complete the following sections with your results, explanations, and conclusions. This includes the abstract and this introduction (i.e., replace its contents!). For this section: introduce what the paper is about and provide a background to any relevant literature (using proper citations, e.g. ~\cite{holland1992genetic}).

\section{Literature Study}
Search the literature for one paper that applies a Genetic Algorithm to a TSP. This can be from any discipline, i.e., social network analysis, biology, sociology, etc. Minor students are encouraged to take the lead in writing this section.
Summarize the paper, describe the problem, and add any relevant literature to this section. Be
sure to answer at least the following questions about the paper:
\begin{itemize}
    \item How was the algorithm applied? Describe the field and context.
    \item Why did the researchers choose to apply a GA for their application, and were there any
alternatives?
    \item Was their approach successful? Interpret their results.
    \item Give your opinion on their approach. Would you have used the same algorithm in their
situation?
    \item How would you improve on their setup?
\end{itemize}

\section{Problem Description}\label{sect:descr}
The task is to find the shortest possible route to visit all European capital cities, starting and ending in Leiden, Netherlands. Summarize this problem in your own words. You should include a proper (mathematical) problem definition for the TSP in your paper. Be sure to define it to be compatible with the GA you will implement. 

\textbf{Note:} To compute the distance between two cities $c_1$ and $c_2$, we will use the Haversine distance, denoted by $d_H(c_1, c_2)$ in your report. 

\section{Methods}\label{sect:impl}
Give an overview of the two algorithms you implemented, the first being random search and the second the Genetic Algorithm. Be specific for the GA you implemented, and describe
your design choices, e.g., the choice of certain operators, the implementation of a given variant, etc. Do not forget to include references to the work the algorithm initially appeared in. 

Include a subsection on your experimental setup, which should describe the number of runs you did, the maximum number of function evaluations in each experiment, etc. Also, describe which parameter settings you use in your experiments and how this might impact the behavior of your algorithm. 

\section{Results}
In your results, compare the two variants of GA, each with a different mutation operator, to each other and Random Search. In your discussion, focus on the final solution quality (e.g., how good the tour you end up with is) and compare how quickly each method converges to a solution, measured in fitness function evaluations. In other words, you should measure how many evaluations of the TSP each method uses to solve the problem. A great way to visualize this is using a \textit{convergence} plot. 

\section{Conclusion}\label{sec:conclusion}
Write a \textbf{short} conclusion summarizing the most important findings of this assignment. 

\appendix
\section{Appendix}
\subsection{Submission, review and grading}\label{sec:submission}
For this assignment, \textbf{you should submit the full report in this template}. Don’t forget to proofread your report and correctly label and format figures and tables. Please do not exceed the \textbf{limit of 6 pages} for this report, excluding references.

Submission should be done on Brightspace, including your PDF report and your Python file code. Ensure the code is readable: clear variable names, comments, etc. 

\bibliographystyle{splncs04}
\bibliography{bibliography.bib}

\end{document}